\documentclass[dutch]{article}

\usepackage[dutch]{babel}
\usepackage[style=ieee, backref=true, backend=biber]{biblatex}
\addbibresource{referenties.bib} % Importeert het bibliografiebestand

\usepackage{parskip}
\usepackage{doc}
\usepackage{amsmath, amssymb}
\usepackage{color}
\usepackage{qrcode}
\usepackage{graphicx} 
\usepackage{bookmark} 
\usepackage{enumerate}
\usepackage{float}

\usepackage[per-mode=reciprocal, separate-uncertainty=true]{siunitx}
\addto\extrasgerman{\sisetup{locale = DE}}
\sisetup{%
	output-decimal-marker = {,},
	inter-unit-product = \ensuremath{{}\cdot{}},
	exponent-product = \ensuremath{{}\cdot{}},
	list-final-separator = { en },
	list-pair-separator = { en },
	range-phrase = { tot },
}

\usepackage{url}
\usepackage{hyperref}
\hypersetup{
  colorlinks    = true,
  urlcolor      = blue,
  linkcolor     = blue,
  citecolor     = purple,
  bookmarksopen = true,
  bookmarksopenlevel = 2,
}

\usepackage{metalogo}
% \usepackage{dingbat}

\usepackage{tabularray}
\usepackage{caption}
\captionsetup{labelfont={bf}}
\definecolor{bg}{rgb}{0.95,0.95,0.95}

\usepackage{listings}
\lstset{language=C++}
\usepackage{minted} 
\setminted[latex]{obeytabs=true, tabsize=2, breaklines, framesep=\baselineskip, autogobble=true, frame=single}%, bgcolor=bg}
\setminted[bibtex]{obeytabs=true, tabsize=2, breaklines, framesep=5mm, autogobble=true, frame=single}
\setminted[text]{obeytabs=true, tabsize=2, breaklines, framesep=5mm, autogobble=true}
\newminted{cpp}{gobble=2,linenos}
\setmintedinline{bgcolor={}}
\newcommand{\mintinlinecustom}[1]{\mintinline{latex}{#1}}
% \usemintedstyle{bw}

\usepackage[a4paper,left=3cm,right=2.5cm,top=1.5cm,bottom=1.5cm,heightrounded]{geometry}

% \usepackage{tcolorbox}
% \newtcolorbox{mybox_1}{colback=blue!5!white,colframe=blue!75!black}

\usepackage[autostyle=true]{csquotes}

\title{\LaTeX-workshop voor beginners (Handleiding)}
\author{Joost Scheffer, studievereniging Ångström}
\date{\today}
\begin{document}
\maketitle

Als je dit helemaal doorneemt snap je \LaTeX\ echt goed. Dingen overslaan is op eigen risico.
\section{Inleiding}
Dit is de handleiding horende bij de \LaTeX-workshop van studievereniging Ångström\footnote{Deze handleiding is grotendeels gebaseerd op die van studievereniging De Leidsche Flesch \href{https://www.deleidscheflesch.nl/p/latex-workshop}{defles.ch/latex}.}.
Kijk allereerst deze handleiding door.
Ga vervolgens aan de slag met de opdrachten van het werkblad.
Het is de bedoeling dat je bij het maken van deze opdrachten dingen opzoekt in deze handleiding en op internet.

Je zult waarschijnlijk een aantal keren iets fout doen omdat je ergens in je code iets net verkeerd doet.
Dit is bewust de opzet van deze workshop, omdat je later bij het gebruik van \LaTeX\ hier gegarandeerd mee te maken gaat krijgen en het dus belangrijk is om dit op te kunnen lossen.
Onthoud verder ook dat er meestal meerdere manieren zijn om iets weer te geven met telkens een net iets andere lay-out.
Het is vaak een kwestie van wat je fijner vindt.

Mocht het maken van documenten in \LaTeX\ je moeilijk afgaan, probeer dan gebruik te maken van online hulpmiddelen.
Het wikibook over \LaTeX, te vinden op \href{https://en.wikibooks.org/wiki/LaTeX}{en.wikibooks.org/wiki/LaTeX}, is een zeer toegankelijk hulpmiddel waar de meest voorkomende situaties uitgebreid in beschreven staan.
Andere nuttige hulpmiddellen zijn de informatiepagina van de online-editor Overleaf, te vinden op \href{https://www.overleaf.com/learn}{overleaf.com/learn}, en \href{https://www.learnlatex.org/en/}{learnlatex.org/en}.
Mocht je het commando van een bepaald symbool niet kunnen vinden, probeer dan \href{http://detexify.kirelabs.org}{detexify.kirelabs.org}, waar je het symbool dat je zoek kan tekenen, en het bijbehorende commando dan voor je gevonden wordt\footnote{Voor Android is er ook een app beschikbaar \url{https://play.google.com/store/apps/details?id=website.marty.detexify&hl=nl} \qrcode[height=\baselineskip]{https://play.google.com/store/apps/details?id=website.marty.detexify&hl=nl}}.


\section{Wat is \LaTeX}
LaTeX\ (uitspraak: la-tech; traditioneel geschreven als \LaTeX\ met de Griekse letters \textit{tau} en \textit{chi}) is een softwaresysteem en opmaaktaal voor het maken van documenten.
Het is populair in de wetenschappelijke wereld omdat het uitblinkt in het zetten\footnote{\url{https://nl.wikipedia.org/wiki/Zetten_(druktechniek)}} van technische documenten en werken met ingewikkelde documenten vergemakkelijkt.

\LaTeX\ is een opmaaktaal in tegenstelling tot de \textit{what you see is what you get} tekstverwerkers Word en Google Docs.
\LaTeX\ werkt waarschijnlijk anders dan je gewend bent.
In \LaTeX\ schrijf je niet direct je document, maar schrijf je een tekstdocument (\texttt{.tex} bestand) met ``code'' in de opmaaktaal \LaTeX.
Als je van deze ``code'' een PDF wilt maken moet het document ``gecompileerd'' (\textit{compiled} in het Engels) worden, de \LaTeX-engine zorgt er dan voor dat de opmaak overal hetzelfde zal zijn en dat alle verwijzingen naar vergelijkingen of figuren juist worden opgemaakt.
\LaTeX-editors hebben vaak ook twee tabbladen naast elkaar, eentje voor de code van je document en eentje voor de PDF-weergave.
Hoewel dit op het eerste oog heel omslachtig lijkt, blijkt dit voor veel toepassingen heel handig, zeker voor grote documenten, de opmaak van je document kun je centraal instellen met een stijlbestand en er zijn talloze packages om automatisch speciale dingen juist op te maken, idealiter hoef je bij het schrijven van je document niet meer na te denken over de opmaak. Ook is het schrijven van wiskundige vergelijkingen heel makkelijk in \LaTeX.
De voornaamste nadelen zijn dat het met \LaTeX\ veel ingewikkelder is dan je gewend bent om een plaatje te plaatsen waar je wilt en dat je bij grote documenten (meer dan 60 pagina's) soms een tiental seconden moet wachten voordat je kunt zien hoe je document er uit komt te zien.

\subsection{Beginnen}
Voor deze workshop raden we aan de online-editor \href{https://www.overleaf.com}{overleaf.com} te gebruiken, hiermee kun je \textit{in the cloud} werken.
Het werkt in principe zoals Google Docs, aangezien je op elke computer er bij kan en alle bestanden die je maakt ook online opgeslagen worden.
Zo kan je ook nooit je huiswerk meer vergeten.
Ook kun je met meerdere personen aan hetzelfde document werken\footnote{Op het moment van schrijven heeft De Haagse geen Overleaf licentie, een gratis Overleaf account heeft beperkingen. Zo kan er maar met 2 mensen tegelijk in een document gewerkt worden, zal compileren langer duren en zijn er limieten aan hoe lang je document mag duren om te compileren.}.
Hoe je op Overleaf een document aanmaakt is te vinden op \href{https://www.overleaf.com/learn/how-to/Creating_a_document_in_Overleaf}{overleaf.com/learn/how-to/Creating\_a\_document\_in\_Overleaf}.

Daarnaast zijn er veel verschillende offline editors, zoals MiK\TeX\, \TeX maker, \TeX nicCenter, LyX, VSCode en voor Mac gebruikers is er ook Texifier.
Je kunt op je eigen computer met Windows, Mac of Linux MiK\TeX\ downloaden via \href{http://miktex.org/download}{miktex.org/download}.
MiK\TeX\ bevat zowel een \LaTeX\ package manager als een \LaTeX-editor.

Gebruikers van Linux of Mac hebben mogelijk al een (niet grafische) package manager op hun computer staan in de vorm van \TeX live. Dit kun je testen in de terminal de commando's \verb|texliveonfly --help| en \verb|pdflatex --help|.

\section{Nieuw document}
Een \LaTeX\ document bestaat uit twee delen.
Het eerste deel is de preamble.
De dingen die je hier neerzet gelden voor het hele document, hier kun je bijvoorbeeld de juiste pakketten laden, opties meegeven aan de paketten, het lettertype instellen etc{.}.
Het tweede deel begint bij \mintinlinecustom{\begin{document}} en eindigt bij \mintinlinecustom{\end{document}} hier tussenin staat de inhoud van het document.

Je kunt ook commentaar in je \verb|.tex| bestand zetten, zonder dat dit wordt weergeven in je PDF.
Doe dit door \verb|%| aan het begin van je regel met commentaar te zetten.

\subsection{Preamble}

\begin{enumerate}
	\item Zet bovenaan \mintinlinecustom{\documentclass{article}}. Dit commando geeft aan wat voor soort tekst je gaat schrijven, in dit geval dus \texttt{article}. Daarnaast bepaalt het ook de lettergrootte. Een \texttt{article} is het beste voor bijna elke situatie.

	\item Direct daaronder zet je de packages die je wilt gebruiken. Deze maken het mogelijk om een aantal extra functies toe te voegen aan \LaTeX\ die er niet standaard in zitten. Voeg een package toe aan je document door het commando \mintinlinecustom{\usepackage{packagenaam}} bovenaan het document te typen.

	      Een aantal commands dat vaak in bestanden wordt gebruikt zijn:
	      \begin{enumerate}[1)]
		      \item \mintinlinecustom{\usepackage[a4paper]{geometry}} Deze zorgt dat de afmetingen van je bestand op a4 grootte zijn.
		      \item \mintinlinecustom{\usepackage[english, dutch]{babel}} Deze zorgt voor een goede regelafbreking en goede vertalingen. Hierbij moet je denken aan ``Hoofdstuk'' in plaats van ``Chapter''. Hier wordt ook Engels geladen, omdat je bijvoorbeeld Engelse citaten kunt gebruiken. De laatst                  genoemde taal tussen de blokhaken is de hoofdtaal.
		      \item \mintinlinecustom{\usepackage{parskip}} Deze zorgt voor een mooie weergave van alinea's.
		      \item \mintinlinecustom{\usepackage{amsmath, amssymb}} Dit zijn pakketten die ervoor zorgen dat formules en andere wiskundige dingen mooier worden weergeven.
		      \item \mintinlinecustom{\usepackage{graphicx}} Dit is een pakket om losse plaatjes toe te voegen.
		      \item \mintinlinecustom{\usepackage{enumerate}} Hiermee kan je makkelijk genummerde lijsten maken.
		      \item \mintinlinecustom{\usepackage{url}} Dit zorgt ervoor dat URL's leesbaar worden weergeven.
		      \item \mintinlinecustom{\usepackage{float}} Dit kun je gebruiken om je plaatjes op de gewenste plek te krijgen.
		      \item \mintinlinecustom{\usepackage[hidelinks]{hyperref}} Om een inhoudsopgave te maken waarbij je als je klikt op het punt waar je heen wilt, er vanzelf heen springt.
		      \item \mintinlinecustom{\usepackage{color}} Om tekst in {\color{red}kleur} te kunnen typen.
	      \end{enumerate}

	      Soms doet het zich voor dat een bepaald package intern gebruik maakt van een ander package dat jij ook wilt gebruiken. Het kan zijn dat je daardoor sommige packages in een bepaalde volgorde moet importeren. Dit staat vaak beschreven in de documentatie van de packages, te vinden op \href{https://ctan.org/}{ctan.org}.

	\item Zet in je preamble de volgende commands:
	      \begin{minted}{latex}
\title{titel}
\author{A. Utheur}
\date{\today}
	  \end{minted}
	      het commando \mintinlinecustom{\today} voegt automatisch de datum van vandaag toe.

	\item Als laatste zet je \mintinlinecustom{\begin{document}} en \mintinlinecustom{\end{document}} neer. Tussen deze commando's zet je de tekst in je document.
	\item Compileer het document
\end{enumerate}


\subsection{Tekst in document}

\subsubsection{Titel en auteur}
Zet bovenin je document \mintinlinecustom{\maketitle}. Dit zorgt ervoor dat de gegevens die je in de preamble hebt ingevuld als titel op de goede plek in je document komen te staan. Dus er komt in dit geval een titel, auteur en datum te staan.

\subsubsection{Inhoudsopgave}
Een inhoudsopgave maak je simpel door het commando \mintinlinecustom{\tableofcontents} te typen. Daarna wil je beginnen met een nieuwe pagina, dus daaronder typ je \mintinlinecustom{\newpage}.

\subsubsection{Kopjes}
Een article document in \LaTeX\ maakt gebruik van \verb|section|, \verb|subsection|, \verb|subsubsection| en \verb|paragraph|. Zo is bijvoorbeeld in deze handleiding het kopje ``Nieuw document'' een \textit{section}. Deze maak je door te typen \mintinlinecustom{\section{Naam kopje}}.

Het kopje ``Tekst in document'' is een \textit{subsection}. Deze maak je door te typen \mintinlinecustom{\subsection{Naam kopje}}.

Het kopje ``Kopjes'' is een \textit{subsubsection}. Deze maak je door te typen \mintinlinecustom{\subsubsection{Naam kopje}}
\paragraph{Paragraph}
Dan heb je ook nog een \textit{paragraph}, deze krijg je door het commando \mintinlinecustom{\paragraph{Naam paragraph}} te typen. Dit kopje is een \textit{paragraph}.

Let op dat we in het Nederlands typisch niet het woord sectie of section gebruiken. Een \textit{section} heet afhankelijk van de context een paragraaf of een hoofdstuk\footnote{Je kunt hier meer over lezen op \href{https://www.vlaanderen.be/team-taaladvies/taaladviezen/teksten-schrijven/tekststructuur}{www.vlaanderen.be/team-taaladvies/taaladviezen/teksten-schrijven/tekststructuur}, \href{https://nl.wikipedia.org/wiki/Alinea\#Taalverwarring}{nl.wikipedia.org/wiki/Alinea\#Taalverwarring} en op \href{https://nl.wikipedia.org/wiki/Paragraaf}{nl.wikipedia.org/wiki/Paragraaf}.}.

\subsubsection{Alinea}
Je kunt alinea's maken door een lege regel tussen je tekst te maken of door \mintinlinecustom{\\} aan het einde van je regel te plaatsen.

\subsubsection{Opsommingen en lijsten}
Je kunt een niet-genummerde opsomming maken door het commando \mintinlinecustom{\begin{itemize}} te typen. Eén punt in je opsomming maak je dan door \mintinlinecustom{\item} te typen, achter dit commando typ je een spatie en daarachter de tekst die je in dat punt wil hebben. Aan het einde van je opsomming zet je \mintinlinecustom{\end{itemize}}.
\begin{itemize}
	\item Dit is het eerste punt van de niet-genummerde opsomming
	\item Dit het tweede
	\item Dit het laatste
\end{itemize}

Een genummerde opmaak, zoals eerdere opsommingen in deze handleiding, kun je maken door het commando \mintinlinecustom{\begin{enumerate}} te doen, dan maak je weer nummers door \mintinlinecustom{\item} te typen. Weer eindig je met \mintinlinecustom{\end{enumerate}}. Zonder toevoegingen krijg je dan

\begin{enumerate}
	\item voorbeeld 1
	\item voorbeeld 2
\end{enumerate}

Wil je een andere opmaak of letters die oplopen dan typ je \mintinlinecustom{\begin{enumerate}[a.]}. Dan krijg je dit:

\begin{enumerate}[a.]
	\item voorbeeld 1
	\item voorbeeld 2
\end{enumerate}
Op dezelfde manier kan je ook een opsomming met i'tjes maken door \mintinlinecustom{\begin{enumerate}[i.]}
\begin{enumerate}[i.]
	\item voorbeeld 1
	\item voorbeeld 2
\end{enumerate}

\section{Tekst}
Net zoals met Microsoft Word heb je in \LaTeX\ ook de mogelijkheid tekst in
verschillende vormen te presenteren, hieronder een paar voorbeelden.
\begin{itemize}
	\item \textbf{dik}, typ \mintinlinecustom{\textbf{tekst}} (\texttt{bf} staat voor \textit{bold face})
	\item \textit{cursief}, typ \mintinlinecustom{\textit{tekst}} (\texttt{it} staat voor \textit{italics})
	\item \texttt{typemachine}, typ \mintinlinecustom{\texttt{tekst}}
	\item \textcolor{red}{kleur}, typ \mintinlinecustom{\textcolor{kleur}{tekst}} of \mintinlinecustom{{\color{kleur}tekst}}
	\item \small klein, typ \verb.\small. en de tekst erachter
	\item \large groot, typ \verb.\large.
	\item \Large groter, typ \verb.\Large.
	\item \huge nog groter, typ \verb.\huge.
	\item \Huge allergrootst, typ \verb.\Huge. \normalsize.
\end{itemize}
Typ hierna wel weer \mintinlinecustom{\normalsize} om verder te gaan in normale tekstgrootte of sluit je tekstgedeelte in met accolade's \texttt{\{}\ldots\texttt{\}}.

Als je een commando wilt weergeven gebruik dan \mintinlinecustom{\verb|\section{}|}.

Een overzicht van alle opties is te vinden op \href{https://latexref.xyz/Font-styles.html}{latexref.xyz/Font-styles.html}.

\subsection{Accenttekens}
Pure \LaTeX\ ondersteund soms het niet als je direct een letter met een accentteken in je code zet, bijvoorbeeld de letter \'e.
Hiervoor bestaan verschillende oplossingen.
Je kunt dezelfde letter bereiken door het accent teken aan te geven met een commando \mintinlinecustom{\'e}.
Als je veel van deze letters hebt, kun je ook het volgende in je preamble zetten\footnote{Bij moderne \LaTeX\ distributies is \mintinlinecustom{\usepackage[utf8]{inputenc}} typisch niet nodig omdat deze standaard in wordt geladen.}

\begin{minted}{latex}
  \usepackage[utf8]{inputenc}
  \usepackage[T1]{fontenc}
\end{minted}

Als je ook te maken hebt met bijvoorbeeld Cyrillisch schrift of Chinese tekens dan is het aan te raden om in je \LaTeX-omgeving in te stellen dat je de \LuaLaTeX\ engine/compiler wilt gebruiken.
\LuaLaTeX\ is bedoeld als superset van \LaTeX\ en ondersteund van nature verschillende fonts en de UTF-8 standaard.
Als je \LuaLaTeX\ gebruikt voeg dan het package \texttt{fontspec} toe aan je preamble.


\section{Wiskunde omgeving}
\subsection{Vergelijkingen weergeven}
Wil je formules in je document dan moet je net als bij andere programma's een ``wiskundige omgeving'' (\textit{math environment}) toevoegen aan je document.
Een wiskundige omgeving kan binnen de tekst zijn of op een eigen regel.
Wanneer je een wiskundeomgeving op een eigen zet kun je deze omgeving ook een nummer geven.

Er zijn meerdere manieren om een wiskundeomgeving aan te roepen de drie vaakst gebruikte zijn \textit{inline math}, \textit{display math}\footnote{Technisch gezien is een \textit{display math} omgeving niet gebonden aan het hebben van een eigen regel en kan deze ook in een tekstregel worden aangeroepen om er voor te zorgen dat bijvoorbeeld de regelhoogte aan wordt gepast om breuken volledig uit te kunnen schrijven.} en \texttt{equation}/\texttt{align} (een vorm van \textit{display math}).
Bij \textit{inline math} komen de wiskundige symbolen in de regel te staan, zoals $a^n+b^n=c^n$.
Deze omgeving kun je op verschillende manieren aanroepen.
%\footnote{Om bepaalde redenen werkt gaat het weergeven van wiskunde in koppen soms fout bij oudere \LaTeX\ distributies van voor 2018. Je zult dan het package \textit{fixltx2e} moeten toevoegen om het te laten werken}
De bekendste is het zetten van de wiskundige formules tussen \verb|$|-tekens of binnen \mintinlinecustom{\(...\)}\footnote{Het gebruik van \mintinlinecustom{\(} en \mintinlinecustom{\[} voor het beginnen van ongenummerde math environments wordt aangemoedigd omdat dit de standaarden zijn van \LaTeX\ terwijl \mintinlinecustom{$} en \mintinlinecustom{$$} nog afkomstig zijn van het originele \TeX. Wanneer je \mintinlinecustom{$$} gebruikt kunnen er problemen ontstaan met het zetten van wiskundeformules doordat deze achter de schermen andere opmaakcode gebruiken die niet tot de \LaTeX\ standaard behoord. Een van de problemen is ook je geen duidelijke foutmeldingen krijgt. In het bijzonder ontstaan er vaak problemen bij het gebruik van AMS-\LaTeX\ macros. De \mintinlinecustom{equation*} en standaard \textit{display math} omgevingen zijn functioneel equivalent.}.

De \textit{display math} omgeving wordt gebruikt op de wiskundige vergelijkingen te centreren, zoals
\[ \frac{\hbar^2}{2m}\nabla^2\Psi + V(\mathbf{r})\Psi= -i\hbar \frac{\partial\Psi}{\partial t}\]
Je kunt deze omgeving aanroepen door het tussen de volgende haken te plaatsen \mintinlinecustom{\[...\]} of binnen  een \mintinlinecustom{equation*} omgeving (met sterretje) te zetten met \mintinlinecustom{\begin{equation*}} en \mintinlinecustom{\end{equation*}}, zoals
\[ \int_{-\infty}^{\infty} \frac{\cos x}{x^2 + 1} = \frac{\pi}{e} .\]

Als je wilt dat je \textit{display math} omgeving een nummer krijgt zodat je er later naar kunt verwijzen doe je dit met de \texttt{equation} omgeving (zonder sterretje) door je wiskunde te zetten binnen \mintinlinecustom{\begin{equation}} en \mintinlinecustom{\end{equation}}, zoals
\begin{equation}
	\int_{-\infty}^{\infty} \frac{\cos x}{x^2 + 1} = \frac{\pi}{e} .
\end{equation}

Tot slot hebben we ook nog de mogelijkheid om vergelijkingen met elkaar uit te lijnen.
Dit doe je met de \texttt{align} omgeving.
Deze omgeving werkt net zoals de \texttt{equation} omgeving.
Om aan te geven welk teken je uit wilt lijnen zet je een \texttt{\&} voor dit teken, meestal gebruik je het voor het \verb|=|-teken.
Om aan een nieuwe regel te beginnen gebruik je het commando \mintinlinecustom{\\}.
Dit ziet er zo uit
\begin{align}
	\int_{-\infty}^{\infty} \frac{\cos x}{x^2 + 1} dx & = \frac{\pi}{e}  \\
	\int_{0}^{\infty} \frac{\sin x}{x} dx             & = \frac{\pi }{2}
\end{align}
Dit bereik je dus met de volgende code

\begin{minted}{latex}
\begin{align}
	\int_{-\infty}^{\infty} \frac{\cos x}{x^2 + 1} dx &= = \frac{\pi}{e} \\
	\int_{0}^{\infty} \frac{\sin x}{x} dx &= \frac{\pi }{2}
\end{align}
\end{minted}

Wil je uitlijnen zonder nummers, dan typ je \mintinlinecustom{\begin{align*}} en \mintinlinecustom{\end{align*}}.
Als je zin hebt om te experimenteren, probeer dan uit te zoeken wat er gebeurt als je meerdere \verb|&|'s in dezelfde regel gebruikt.

\subsection{Superscript en subscript}
Bij het werken in en wiskunde omgeving worden subscript (\mintinlinecustom{_}) en superscript (\mintinlinecustom{^}) veel gebruikt. Zo kunnen we bijvoorbeeld de rij $a_1, a_2, \dots, a_{n-1}$ weergeven met de code \mintinlinecustom{a_1,a_2,\dots,a_{n-1}} en de verglijking $a^b \cdot a^c = a^{b + c}$ weergeven met de code \mintinlinecustom{a^b \cdot a^c = a^{b + c}}. Let wel op dat bij meer dan \'e\'en teken in super- of subscript dat er \verb|{| en \verb|}| om heen staat.
Als je niet wilt dat je tekst schuin gedrukt is, gebruik dan \mintinlinecustom{\text{...}} in je subscript.

\subsection{Haakjes}

Als we haken om grote symbolen in \textit{display math} zetten, zoals bijvoorbeeld om een breuk heen, is het belangrijk de haakjes te schalen naar het symbool.
Dit kan worden gedaan door de \mintinlinecustom{\left} en \mintinlinecustom{\right} commandos toe te voegen aan de haakjes.
\[\left| -\frac{1}{2} \right| ^3 = \left( \frac{1}{8} \right)\]
wordt geproduceerd met de volgende code:
\begin{minted}{latex}
\[\left| -\frac{1}{2} \right| ^3 = \left( \frac{1}{8} \right)\]
\end{minted}

Als je een linkerhaakje wilt laten matchen met een rechterhaakje op een andere regel kan je de onzichtbare \mintinlinecustom{\right.} gebruiken. Dit komt bijvoorbeeld voor bij lange definities.
\[\left.\text{s.t. } f^2(x) < \frac{\epsilon}{2} \right\} \]
wat wordt geproduceerd door de code
\begin{minted}{latex}
\[\left.\text{s.t. } f^2(x) < \frac{\epsilon}{2} \right\} \]
\end{minted}
(Let op het ``escapen'' van de accolade).

\subsection{Wortels en breuken}
Bij het gebruik van wiskunde zul je ook veel breuken of wortels gebruiken. Voor het weergeven van de breuk $\frac{a}{b}$ typ je in je code \mintinlinecustom{\frac{a}{b}} en voor het weergeven van de wortel $\sqrt{a}$ gebruiken we \mintinlinecustom{\sqrt{a}} of voor $\sqrt[n]{a}$ gebruiken we \mintinlinecustom{\sqrt[n]{a}}

\subsection{Speciale tekens}
Er zijn veel symbolen die bij wiskunde gebruikt kunnen worden. In deze paragraaf zullen een aantal worden genoemd, maar er zijn er veel meer. Om snel op te kunnen zoeken welke code bij een teken hoort kun je de website of app Detexify\footnote{\url{http://detexify.kirelabs.org/}} raadplegen. Hierbij kun je denken aan integralen, maar ook aan bijvoorbeeld $\alpha$ en $\beta$. Hieronder staan de meest gebruikte symbolen uit het griekse alfabet.
\begin{center}
	\begin{tabular}{| l | l | l | l | l | l | l | l | l | l | l | l | }
		\hline
		         & code          &          & code          &           & code           &           & code           &            & code            &               & code               \\
		\hline
		$\alpha$ & \verb|\alpha| & $\beta$  & \verb|\beta|  & $\gamma$  & \verb|\gamma|  & $\delta$  & \verb|\delta|  & $\epsilon$ & \verb|\epsilon| & $\varepsilon$ & \verb|\varepsilon| \\
		\hline
		$\zeta$  & \verb|\zeta|  & $\theta$ & \verb|\theta| & $\lambda$ & \verb|\lambda| & $\mu$     & \verb|\mu|     & $\xi$      & \verb|\xi|      & $\pi$         & \verb|\pi|         \\
		\hline
		$\rho$   & \verb|\rho|   & $\sigma$ & \verb|\sigma| & $\tau$    & \verb|\tau|    & $\phi$    & \verb|\phi|    & $\varphi$  & \verb|\varphi|  & $\chi$        & \verb|\chi|        \\
		\hline
		$\psi$   & \verb|\psi|   & $\Gamma$ & \verb|\Gamma| & $\Delta$  & \verb|\Delta|  & $\Lambda$ & \verb|\Lambda| & $\Phi$     & \verb|\Phi|     & $\Psi$        & \verb|\Psi|        \\
		\hline
	\end{tabular}
\end{center}

Naast tekens uit het Griekse alfabet hebben we ook nog standaard functies, waarvoor commando's in \LaTeX\ bestaan. Deze standaard functies worden op de goede manier aangegeven als het commando wordt gebruikt. De meest gebruikte hiervan zijn:
\begin{center}
	\begin{tabular}{| l | l | l | l | l | l | l | l | l | l | l | l |}
		\hline
		          & code           &           & code           &           & code           &        & code        &         & code         &        & code        \\
		\hline
		$\arccos$ & \verb|\arccos| & $\arcsin$ & \verb|\arcsin| & $\arctan$ & \verb|\arctan| & $\arg$ & \verb|\arg| & $\sinh$ & \verb|\sinh| & $\cos$ & \verb|\cos| \\
		\hline
		$\cosh$   & \verb|\cosh|   & $\deg$    & \verb|\deg|    & $\det$    & \verb|\det|    & $\dim$ & \verb|\dim| & $\tan$  & \verb|\tan|  & $\ker$ & \verb|\ker| \\
		\hline
		$\lim$    & \verb|\lim|    & $\log$    & \verb|\log|    & $\max$    & \verb|\max|    & $\min$ & \verb|\min| & $\sin$  & \verb|\sin|  &        &             \\
		\hline
	\end{tabular}
\end{center}

Naast de tekens uit het Griekse alfabet en de standaard functies hebben we ook nog enkele sommen, producten, integralen etc. Deze symbolen kunnen zich aanpassen naar de omgeving waarin zij zich bevinden. Zo worden zij groter weergeven in \textit{display math} dan in \textit{math}. De symbolen hieronder worden het meest gebruikt:
\begin{center}
	\begin{tabular}{| l | l | l | l | l | l | l | l |}
		\hline
		        & code         &           & code           &          & code          &           & code           \\
		\hline
		$\sum$  & \verb|\sum|  & $\bigcap$ & \verb|\bigcap| & $\iint$  & \verb|\iint|  & $\bigcup$ & \verb|\bigcup| \\
		\hline
		$\prod$ & \verb|\prod| & $\int$    & \verb|\int|    & $\iiint$ & \verb|\iiint| & $\oint$   & \verb|\oint|   \\
		\hline
	\end{tabular}
\end{center}

Bij zowel $\sum$ als $\prod$ als $\lim$ is het netjes als in \textit{display math} omgeving de boven- en ondergrenzen direct boven of onder de kwantor staat. Het gebruik van subscript en superscript. Zo krijg je bijvoorbeeld
\[\sum_{n=1}^{\infty} \frac{1}{n}\]
door de volgende code \mintinlinecustom{\sum_{n=1}^{\infty} \frac{1}{n}} te gebruiken. Bij bijvoorbeeld dubbele integralen is het netjes om het commando \mintinlinecustom{\limits} te gebruiken, dit zorgt er voor dat de grenzen op de juiste plek komen te staan. Vergelijk bijvoorbeeld:
\[\iint_D \text{ met } \iint\limits_D\]
verkregen door de code \mintinlinecustom{\[\iint_D \text{ met } \iint\limits_D\]}, hierbij is de tweede correct.
Tot slot hebben we nog een aantal tekens die handig zijn om te weten.
\begin{center}
	\begin{tabular}{| l | l | l | l | l | l | l | l |}
		\hline
		               & code                &                   & code                   &                   & code                   &                   & code                   \\
		\hline
		$\pm$          & \verb|\pm|          & $\times$          & \verb|\times|          & $\setminus$       & \verb|\setminus|       & $\cdot$           & \verb|\cdot|           \\
		\hline
		$\mp$          & \verb|\mp|          & $\gg$             & \verb|\gg|             & $\ll$             & \verb|\ll|             & $\top$            & \verb|\top|            \\
		\hline
		$\leq$         & \verb|\leq|         & $\in$             & \verb|\in|             & $\subset$         & \verb|\subset|         & $\subseteq$       & \verb|\subseteq|       \\
		\hline
		$\cong$        & \verb|\cong|        & $\geq$            & \verb|\geq|            & $\approx$         & \verb|\approx|         & $\mid$            & \verb|\mid|            \\
		\hline
		$\equiv$       & \verb|\equiv|       & $\sim$            & \verb|\sim|            & $\neq$            & \verb|\neq|            & $\leftarrow$      & \verb|\leftarrow|      \\
		\hline
		$\Leftarrow$   & \verb|\Leftarrow|   & $\leftrightarrow$ & \verb|\leftrightarrow| & $\Leftrightarrow$ & \verb|\Leftrightarrow| & $\mapsto$         & \verb|\mapsto|         \\
		\hline
		$\iff$         & \verb|\iff|         & $\rightarrow$     & \verb|\rightarrow|     & $\Rightarrow$     & \verb|\Rightarrow|     & $ \updownarrow$   & \verb|\updownarrow|    \\
		\hline
		$\Updownarrow$ & \verb|\Updownarrow| & $\uparrow$        & \verb|\uparrow|        & $\Uparrow$        & \verb|\Uparrow|        & $\downarrow$      & \verb|\downarrow|      \\
		\hline
		$\Downarrow$   & \verb|\Downarrow|   & $\cdots$          & \verb|\cdots|          & $\vdots$          & \verb|\vdots|          & $\ddots$          & \verb|\ddots|          \\
		\hline
		$\forall$      & \verb|\forall|      & $\exists$         & \verb|\exists|         & $\infty$          & \verb|\infty|          & $\nabla$          & \verb|\nabla|          \\
		\hline
		$\partial$     & \verb|\partial|     & $\ni$             & \verb|\ni|             & $\supseteq$       & \verb|\supseteq|       & $\Longrightarrow$ & \verb|\Longrightarrow| \\
		\hline
		\dots          & \verb|\dots|        & \ldots            & \verb|\ldots|          &                   &                        &                   &                        \\
		\hline
	\end{tabular}
\end{center}

Voor vergelijkende operatoren zoals $\leq$, $\in$ en $\subset$ kun je ook een onkennende operator maken door \mintinlinecustom{\not} voor de operator te zetten. Dan krijg je bijvoorbeeld $\not\leq$, $\not\in$ en $\not\subset$.

\section{Tabellen en afbeeldingen}
Het toevoegen van tabellen en afbeeldingen werkt op een soortgelijke manier.
Het grootste verschil is dat een tabel typpisch uitgeschreven wordt en een afbeelding ingeladen wordt.
In deze paragraaf leer je eerst hoe je een tabel maakt, daarna hoe je een afbeelding inlaad en daarna hoe je tabellen en afbeeldingen netjes op de pagina kan zetten en een bijschrift geeft.

\subsection{Tabellen maken}
Tabellen kunnen op berschillende manieren gemaakt worden.
De meest gebruikte zijn de tabel (\mintinlinecustom{\begin{tabular}[positie]{opmaak}}) en de array (\mintinlinecustom{\begin{array}[positie]{opmaak}}) omgevingen.
Een array kan alleen worden aangemaakt in een wiskundeomgeving.

De elementen in een rij worden gescheiden met een \mintinlinecustom{&} en je sluit een rij af met \mintinlinecustom{\\}.
Om een horizontale lijn in je tabel te krijgen plaats je \mintinlinecustom{\hline} in de code van je tabel.

Bij het bepalen van de opmaak, bepaal je ook het aantal kolommen waaruit de tabel bestaat. Om te bepalen hoe deze kolommen eruit zien kunnen de volgende commando's worden aangeroepen:
\begin{center}
	\begin{tabular}{| l | l |}
		\hline
		Commando            & Resultaat                                                  \\
		\hline
		\verb|l|            & Voegt een kolom toe waar de tekst links wordt uitgelijnd.  \\
		\verb|r|            & Voegt een kolom toe waar de tekst rechts wordt uitgelijnd. \\
		\verb|c|            & Voegt een kolom toe waar de tekst wordt gecentreerd.       \\
		\verb|p{x}|         & De tekst wordt uitgevuld over een breedte \verb|x|.        \\
		\verb|*{x}{opmaak}| & Herhaalt de structuur in \verb|opmaak| \verb|x| keer.      \\
		$\mid$              & Voegt een verticale lijn toe.                              \\
		$\mid\mid$          & Voegt een dubbele verticale lijn toe.                      \\
		\verb|@{text}|      & Gebruikt \verb|text| als verticale lijn.                   \\
		\hline
	\end{tabular}
\end{center}

%
% Voor het bepalen van de positie kunnen de volgende commando's worden aangeroepen:
% \begin{center}
% 	\begin{tabular}{| l | l |}
% 		\hline
% 		Commando & Resultaat                                                                 \\
% 		\hline
% 		\verb|t| & De tabel is met zijn top gealigneerd op de onderkant van de regel.        \\
% 		\verb|b| & De tabel is met zijn onderkant gealigneerd aan de onderkant van de regel. \\
% 		\verb|c| & De tabel is gecentreerd rond de onderkant van de regel.                   \\
% 		\hline
% 	\end{tabular}
% \end{center}


Voor het maken van tabellen zijn er verschillende hulpmiddelen beschikbaar, bijvoorbeeld \href{https://www.tablesgenerator.com}{tablesgenerator.com}.

\subsection*{Voorbeeld}
\begin{tabular}{| l | c | r |}
	\hline
	Tijd in maanden & Aantal konijnen & Berekening \\
	\hline
	0               & 1               & 1          \\
	1               & 2               & 1+1        \\
	2               & 3               & 1+2        \\
	3               & 5               & 2+3        \\
	4               & 8               & 3+5        \\
	5               & 13              & 5+8        \\
	6               & 21              & 8+13       \\
	$\vdots$        & $\vdots$        & $\vdots$   \\
	\hline
\end{tabular}

Deze tabel kan worden aangeroepen met de volgende code:
% \begin{verbatim}
\begin{minted}{latex}
\begin{tabular}[c]{| l | c | r |}
\hline
Tijd in maanden & Aantal konijnen & Berekening \\
\hline
0 & 1 & 1 \\
1 & 2 & 1+1  \\
2 & 3 & 1+2 \\
3 & 5 & 2+3 \\
4 & 8 & 3+5 \\
5 & 13 & 5+8 \\
6 & 21 & 8+13 \\
$\vdots$ & $\vdots$ & $\vdots$ \\
\hline
\end{tabular}
\end{minted}

\subsection*{Tabellen verder opmaken}
Voor het opmaken van tabellen zijn er verschillende packages beschikbaar.
Het moderne (sinds 2021) package \texttt{tabularray} is misschien wel het eenvoudigst in gebruik.
Zo kun je het volgende tabel\\
\definecolor{GRIJS}{RGB}{34, 51, 67}
\begin{tblr}[]{
		colspec = {cccc},
		colsep = 1.9ex,
		vline{2-Y} = {GRIJS, solid, 1pt},
		vline{2-Y} = {2-Z}{white, solid, 1pt},
		row{odd} = {GRIJS!20},
		row{even} = {white},
		row{1} = {GRIJS,fg=white,font=\bfseries}
	}
	Tijd in maanden & Aantal konijnen & Berekening \\
	0               & 1               & 1          \\
	1               & 2               & 1+1        \\
	2               & 3               & 1+2        \\
	3               & 5               & 2+3        \\
	4               & 8               & 3+5        \\
	5               & 13              & 5+8        \\
	6               & 21              & 8+13       \\
	$\vdots$        & $\vdots$        & $\vdots$
\end{tblr}\\

maken met de volgende code:

\begin{minted}{latex}
\definecolor{GRIJS}{RGB}{34, 51, 67}
\begin{tblr}[]{
		colspec = {cccc},
		colsep = 1.9ex,
		vline{2-Y} = {GRIJS, solid, 1pt},
		vline{2-Y} = {2-Z}{white, solid, 1pt},
		row{odd} = {GRIJS!20},
		row{even} = {white},
		row{1} = {GRIJS,fg=white,font=\bfseries}
	}
	Tijd in maanden & Aantal konijnen & Berekening \\
	0 & 1 & 1 \\
	1 & 2 & 1+1  \\
	2 & 3 & 1+2 \\
	3 & 5 & 2+3 \\
	4 & 8 & 3+5 \\
	5 & 13 & 5+8 \\
	6 & 21 & 8+13\\
  $\vdots$ & $\vdots$ & $\vdots$ 
\end{tblr}
\end{minted}

\subsection{Afbeeldingen invoegen}
Het is ook mogelijk om afbeeldingen in te voegen bij \LaTeX. Hiervoor heb je het moet je bovenaan je code \mintinlinecustom{\usepackage{graphicx}} zetten.
Vervolgens gebruik je de code\\ \mintinlinecustom{\includegraphics{naam_afbeelding.jpg}} om de afbeelding ook daadwerkelijk in te voegen.
Om dit te doen moet de afbeelding die je wilt invoegen ook in de map staan waarin je \LaTeX-code ook staat.
Let op: \LaTeX\ kan alleen \verb|.pdf|-, \verb|.png|- of \verb|.jpg|-bestanden lezen.

Om het formaat van de afbeelding aan te passen kun je de code\\ \mintinlinecustom{\includegraphics[scale=0.5]{naam_afbeelding.jpg}}. Hierbij geef je de schaal van de afbeelding, in plaats hiervan kun je ook \mintinlinecustom{[width=0.9\linewidth]} gebruiken. Zo wordt de afbeelding ook aangepast naar de gewenste grootte.

\subsection{Tabellen en afbeeldingen zetten}
Om afbeeldingen en tabellen netjes op je pagina te plaatsen kan een ``drijvende'' \textit{float} omgeving gebruikt worden.
Voor tabellen bestaat hiervoor de \texttt{table} omgeving en voor figuren de \texttt{figure} omgeving.
Bij een float omgeving gaat \LaTeX\ tijdens compileren van je document proberen de float op de best mogelijke plek te plaatsen.
Bij het aanroepen van een float omgeving kun je opties meegeven om \LaTeX\ te laten weten waar je wilt dat je float terrecht komt.

Hieronder vind je welke commando's handig zijn om te weten.
\begin{center}
	\begin{tabular}{| l | p{0.8\linewidth} |}
		\hline
		commando & resultaat                                                                                           \\
		\hline
		h        & Probeer de omgeving hier te plaatsen.                                                               \\
		\hline
		t        & Probeer de omgeving aan de bovenkant van een pagina te plaatsen.                                    \\
		\hline
		b        & Probeer de omgeving aan de onderkant van een pagina te plaatsen.                                    \\
		\hline
		p        & Probeer de omgeving op een lege pagina te plaatsen samen met andere omgevingen.                     \\
		\hline
		!        & Maak de opmaakregels iets minder streng waardoor de omgeving makkelijker op een gewenste plek komt. \\
		\hline
		H        & (indien \texttt{float} package is ingeladen) Forceer de omgeving hier.                              \\
		\hline
	\end{tabular}
\end{center}
Let wel op bij gebruik van de H parameter, deze kan er voor zorgen dat de floats niet meer de bedoelde volgorde aanhouden en veroorzaakt vaak een lelijke opmaak.

Een code om een afbeeling te plaatsen ziet er vaak zo uit:
\newpage
\begin{minted}{latex}
\begin{figure}[htpb!]
\centering
\includegraphics[width=0.5\textwidth]{naam_afbeelding.jpg}
\caption{Bijschrift bij afbeelding}
\end{figure}
\end{minted}

Zoals je in de code hierboven kunt zien, is het ook mogelijk om een bijschrift te hebben bij je drijvende omgeving. Dit kun je doen met het commando \mintinlinecustom{\caption{beschrijving}}.
Let op dat bij tabellen het bijschrift typisch boven de tabel staat.

Bij deze drijvende omgeving komt het bijschrift onder de afbeelding of tabel te staan. Het is ook mogelijk om het bijschrift naast de afbeelding te zetten. Dit kun je doen met het \texttt{sidecap} package en de volgende code:
\begin{minted}{latex}
\begin{SCfigure}
\caption{Korte informatie over de afbeelding}
\includegraphics[width=0.5\textwidth]{naam_afbeelding.jpg}
\end{SCfigure}
\end{minted}

Verder is het ook mogelijk om de tekst om de afbeelding heen te laten lopen. Zo komt de afbeelding in de tekst te staan. Hiervoor moet je allereerst bovenaan je code het volgende package aanroepen: \mintinlinecustom{\usepackage{wrapfig}}. Om dan vervolgens de foto te plaatsen gebruik je de volgende code:
\begin{minted}{latex}
	\begin{wrapfigure}{r}{0.5\textwidth}
		\includegraphics[width=0.45\textwidth]{naam_afbeelding.jpg}
		\caption{Korte informatie over de afbeelding}
	\end{wrapfigure}
\end{minted}

Hierbij kun je tussen de tweede accolades, waar in het voorbeeld een \verb|r| staat, de plaatsing van de afbeelding bepalen. Hieronder vind je de meest handige commando's om te weten voor deze plaatsing:
\begin{center}
	\begin{tabular}{| l | l |}
		\hline
		commando & resultaat                                                                               \\
		\hline
		r        & Langs de rechterkant.                                                                   \\
		\hline
		l        & Langs de linkerkant.                                                                    \\
		\hline
		R        & Langs de rechterkant met een drijvende optie, om splitsing over pagina's tegen te gaan. \\
		\hline
		L        & Langs de linkerkant met drijvende optie.                                                \\
		\hline
	\end{tabular}
\end{center}

Tot slot is het ook mogelijk om verschillende afbeeldingen naast elkaar of onder elkaar te plaatsen. Dit kan fijn zijn om verschillende grafieken de kunnen vergelijken. Hiervoor moet je allereerst bovenaan je code het volgende package aanroepen: \mintinlinecustom{\usepackage{subfig}}. Vervolgens kun je de volgende code gebruiken om een blok van vier afbeeldingen te maken. Hierbij komen twee afbeeldingen rijen van twee afbeeldingen te staan.

\begin{minted}{latex}
\begin{figure}
	\centering
	\subfigure[Beschrijving 1]{\includegraphics[width=0.4\textwidth]{grafiek1}
	~
	\subfigure[Beschrijving 2]{\includegraphics[width=0.4\textwidth]{grafiek2} \\
	\subfigure[Beschrijving 3]{\includegraphics[width=0.4\textwidth]{grafiek3}
	~
	\subfigure[Beschrijving 4]{\includegraphics[width=0.4\textwidth]{grafiek4}
	\caption{Grafieken van ...}
\end{figure}
\end{minted}



\section{Code invoegen}

Het invoegen van code kan op verschillende manieren.
Het meest eenvoudige is de ingebouwde \texttt{verbatim} omgeving.

Het package \texttt{listings} bied syntax-ondersteuning aan.
Welke programmeertaal je gebruikt geef je als volgt aan; als je bijv. C++ gebruikt zet je
\mintinlinecustom{\lstset{language=C++}} in je preamble.
Hierna kun je de \texttt{lstlisting} omgeving gebruiken.

De \LaTeX-code
\begin{minted}{latex}
\begin{lstlisting}
  int main (int argc, char** argv) {
    std::cout << "Hello world" << std::endl;
    return 0;
  }
\end{lstlisting}
\end{minted}
resulteert in
\begin{lstlisting}
  int main (int argc, char** argv) {
    std::cout << "Hello world" << std::endl;
    return 0;
  }
\end{lstlisting}

Bovendien kun je simpelweg het bestand zelf invoegen door
\begin{minted}{latex}
\lstinputlisting{bestandnaam.extensie}
\end{minted}
te gebruiken.

Voor uitgebreidere opmaak die soortgelijk is aan moderne editors kan het pacakge \texttt{minted} gebruikt worden.
Dan kun je het volgende commando gebruiken om bestanden in te laden
\begin{minted}{latex}
\inputminted[breaklines, linenos]{Matlab}{code/mijn_matlab_programma.m}
\end{minted}

De \LaTeX-code
% \begin{verbatim}
\inputminted{latex}{matlabvoorbeeld.tex}
resulteert in
\definecolor{bg}{rgb}{0.95,0.95,0.95}
\begin{minted}[breaklines, framesep=\baselineskip, linenos, bgcolor=bg]{matlab}
function [ result ] = fibonacci( n )

if n==0||n==1
    result = n;

else
    result = fibonacci(n-2)+fibonacci(n-1);
end
end
\end{minted}

\section{Verwijzingen en referenties}
In \LaTeX\ kun je automatisch refereren naar eigenlijk alle genummerde dingen.
Bij eigenlijke alle genummerde dingen kun het commando \mintinlinecustom{\label{HierEenNaam}} gebruiken.
Als je \mintinlinecustom{\label{...}} eenmaal op de juiste plaats gezet hebt, kun je het nummer gebruiken met \mintinlinecustom{\ref{HierEenNaam}}.
Ook kun je met \mintinlinecustom{\pageref{HierEenNaam}} verwijzen naar de pagina waar de label staat.
Typpisch geeft met in het label aan wat voor iets het is, een figuur krijgt dan het label \texttt{fig:opstelling} en een tabel \texttt{tab:fibonacci}.

Als je wilt dat je verwijzingen automatisch de verwijstekst bevatten kun je het \texttt{cleverref} pacakge gebruiken. Het commando \mintinlinecustom{\cref{fig:kikker}} resulteert dan in de tekst ``Figuur 4'' in plaats van alleen het getal ``4''.
Het package kun je bij het importeren ook de opties \texttt{capitalise}, \texttt{dutch}, en \texttt{noabbrev} om de verwijzing een hoofdletter te geven, Nederlands te laten gebruiken en geen afkortingen te laten gebruiken.

Voor een voetnoot in je tekst gebruik je \mintinlinecustom{\footnote{Dit is een voetnoot}}\footnote{Dit is een voetnoot}.

\subsection{Bronvermelding}
Als het gaat om bronvermeldingen in \LaTeX, zijn er drie populaire opties: bibtex, natbib en biblatex.

Biblatex is een moderne optie voor het verwerken van bibliografische informatie, biedt een eenvoudiger en flexibeler interface en een betere taal-localisatie dan de andere twee opties.

Een minimaal werkend voorbeeld van een document met bronvermelding ziet er als volgt uit:
\begin{minted}{latex}
\documentclass[a4paper,10pt]{article}
\usepackage{biblatex} % Importeert biblatex pakket
\addbibresource{referenties.bib} % Importeert het bibliografiebestand

\begin{document}
Laten we citeren! Het tijdschriftartikel van Einstein \cite{einstein} en het boek van Dirac \cite{dirac} zijn items gerelateerd aan de natuurkunde.

\printbibliography % Drukt bibliografie af

\end{document}
\end{minted}
In dit voorbeeld komen de volgende commando's voor:
\begin{itemize}
	\item \mintinlinecustom{\usepackage{biblatex}} Importeert biblatex.
	\item \mintinlinecustom{\addbibresource{referenties.bib}} Importeert het databestand van de bibliografie \texttt{referenties.bib}: dit bestand bevat informatie over elk geciteerd boek, artikel, enz. Zie het gedeelte over het bibliografiebestand voor meer informatie.
	\item \mintinlinecustom{\cite{einstein}} Deze opdracht voegt een referentie in het document in, in dit geval [1], die overeenkomt met een element in de bibliografie. 'einstein' is een \textit{keyword} dat overeenkomt met een item in \texttt{referenties.bib}. Dit ziet er zo uit: \cite{einstein}.
	\item \mintinlinecustom{\printbibliography} Drukt de lijst met geciteerde referenties af, de standaardtitel is ``Referenties'' voor de artikel documentklasse en ``Bibliografie'' voor boeken en rapporten.
\end{itemize}

Het bestand \texttt{sample.bib} zou er als volgt uitzien:
\begin{minted}{bibtex}
@article{einstein,
  title = {Über Einen Die {{Erzeugung}} Und {{Verwandlung}} Des {{Lichtes}} Betreffenden Heuristischen {{Gesichtspunkt}}},
  author = {Einstein, A.},
  date = {1905-01},
  journaltitle = {Annalen der Physik},
  shortjournal = {Annalen der Physik},
  volume = {322},
  number = {6},
  pages = {132--148},
  issn = {0003-3804, 1521-3889},
  doi = {10.1002/andp.19053220607},
  url = {https://onlinelibrary.wiley.com/doi/10.1002/andp.19053220607},
  urldate = {2024-01-07},
  langid = {english},
}
\end{minted}
Hier is \texttt{@article} het \textit{entry type} van deze bron, \texttt{einstein} het sleutelwoord en zijn de velden \texttt{title}, \texttt{author} etc{.} invoervelden.

Naast article bestaan er nog meer \textit{entry types}, de belangrijkste zijn \texttt{@book}, \texttt{@online} en \texttt{@manual}.

Een boek ziet er zo uit in een \texttt{.bib} bestand:
\begin{minted}{bibtex}
@book{giancoliNatuurkundeDeelMechanica2022,
  title = {Natuurkunde / Deel 1, Mechanica en thermodynamica / Nederlandstalige bewerking: Dirk Poelman, Jan Ryckebusch ; vertaling [uit het Engels]: Josefien Bruijn},
  shorttitle = {Natuurkunde / Deel 1, Mechanica en thermodynamica / Nederlandstalige bewerking},
  author = {Giancoli, Douglas C.},
  editor = {Poelman, Dirk and Ryckebusch, Jan},
  translator = {Bruijn, Josefien},
  date = {2022},
  edition = {Vijfde editie},
  publisher = {{Pearson}},
  location = {{Amsterdam}},
  isbn = {978-90-430-3723-5},
  langid = {dutch},
  annotation = {OCLC: 1311495422}
}
\end{minted}

De twee bronnen die hier zijn weergeven bevatten veel optionele velden.
Afhankelijk van de stijl die je bronvermelding gebruikt zijn veel velden niet nodig.

Typpisch hebben de websites de uitgevers van een artikel een ``cite'' knop die je in staat stelt te exporteren naar een \texttt{.bib} entry\footnote{Bijvoorbeeld \url{https://www.bibtex.com/converters/}}.
Om jezelf het leven makkelijker te maken zijn er ook websites die DOI's en ISBN-nummers voor je omzetten in een \texttt{.bib} entry. Dit kun je ook doen met een \textit{reference manager} (bijvoorbeeld Zotero of Mendeley) in combinatie met een webbrowser plugin, als je die hebt kun je bronnen simpelweg aan je database toevoegen door in je browser te klikken op een knopje. Reference managers ondersteunen vaak ook ophalen van informatie op basis van een DOI of ISBN-nummer en stellen je in staat om je database te exporteren naar een \texttt{.bib} bestand.

Als je toch handmatig dingen wilt toevoegen kun je het BibLaTeX-cheatsheet gebruiken, te vinden op \href{https://tug.ctan.org/info/biblatex-cheatsheet/biblatex-cheatsheet.pdf}{tug.ctan.org/info/biblatex-cheatsheet/biblatex-cheatsheet.pdf}.

Als je handmatig compileert of een editor gebruikt die maar één keer compileert kan het zijn dat er op de plaats van je verwijzingen vraagtekens komen te staan. \textbf{Voor verwijzingen is het belangrijk om de code 2x te compileren. Anders komen er vraagtekens of verouderde nummering te staan.}

Let ook op dat \texttt{.bib} bestanden erg gevoelig zijn voor het ontbreken van een komma.

BiBLaTeX ondersteund ook opties, zo zet je bijvoorbeeld IEEE-stijl aan met de optie \texttt{style=ieee} en is het voor grote documenten handig om de optie \texttt{backref=true} toe te voegen. Het importeren ziet er dan zo uit:
\begin{minted}{latex}
\usepackage[style=ieee, backref=true, backend=biber]{biblatex}
\end{minted}

\section{Nederlandse notatie en eenheden}
\LaTeX\ zet automatisch een witruimte achter elke komma, dit is ongewenst wanneer je in het Nederlands een getal noteerd. Het package \texttt{icomma} importeren lost dit op.

Om eenheden en getallen automatisch goed te laten gaan kan het package \texttt{siunitx} gebruikt worden met de volgende opties:
\begin{minted}{latex}
\usepackage[per-mode=reciprocal, separate-uncertainty=true]{siunitx}
\addto\extrasgerman{\sisetup{locale = DE}}
\sisetup{%
	output-decimal-marker = {,},
	inter-unit-product = \ensuremath{{}\cdot{}},
	exponent-product = \ensuremath{{}\cdot{}},
	list-final-separator = { en },
	list-pair-separator = { en },
	range-phrase = { tot },
}
\end{minted}
Wanneer je een meting aan wilt geven kan dit met de commando's \mintinlinecustom{\qty}, \mintinlinecustom{\num}, \mintinlinecustom{\unit} en \mintinlinecustom{\qtyrange}.

\begin{itemize}
	\item \mintinlinecustom{\qty{50+-10}{\metre\per\second\squared}} wordt vanzelf opgemaakt in \qty{50+-10}{\metre\per\second\squared}
	\item \mintinlinecustom{\num{10.1234}} wordt vanzelf opgemaakt in \num{10.1234}
	\item \mintinlinecustom{\unit{\square\volt\cubic\lumen\per\farad}} wordt vanzelf opgemaakt in \unit{\square\volt\cubic\lumen\per\farad}
	\item \mintinlinecustom{\qtyrange{20}{30}{\volt}} wordt vanzelf opgemaakt in \qtyrange{20}{30}{\volt}
\end{itemize}

Om aanhalingstekens in het Nederlands juist te laten verlopen is het package \mintinlinecustom{csquotes} nodig.

\section{Overige interessante packages}
Verder zijn er nog allerlei verschillende packages om je leven makkelijker te maken. Zoek ze eens op op internet!
\begin{itemize}
	\item \verb&\usepackage{fancyhdr}& Voor intelligente headers en footers op je pagina.
	\item \verb&\usepackage[section]{placeins}& Zorg ervoor dat floating objects niet voorbij een sectieovergang gaan. Daarnaast kan je met \verb&\FloatBarrier& handmatig een barriere defini\"eren voor alle floats.
	\item \texttt{latexmk} Kan het hele compileerproces voor je automatiseren.
	\item \verb&\usepackage{sidecap}& Voor captions naast je floats in plaats van onder of boven.
	\item \verb&\usepackage{beamer}& Om beamer-presentaties te maken (powerpoint).
	\item Bij het package \texttt{geometry} kun je de optie \texttt{showframe} meegeven om de paginamarges te zien.
	\item \verb&\usepackage{tikzducks}& Om parametrisch eenden toe te voegen aan je documenten.
	\item Wist je dat \verb&\\&, het commando voor een regelovergang, een optioneel argument heeft? Probeer bijvoorbeeld eens \verb&\\[5cm]& .
	\item Er is een \LuaLaTeX\ package om aan te geven of een artikel open-acces is. \texttt{biblatex-ext-oa} zoekt automatisch op het internet voor welke bronnen er een open-acces versie beschikbaar is en voegt deze toe.
	\item Wist je dat je voor de documentatie van een package naar \href{https://ctan.org/pkg}{ctan.org/pkg} kunt? Pas wel op want sommige grote packages hebben meer dan duizend pagina's aan documentatie.
	\item Wist je dat je een reference manageer kunt gebruiken om eenvoudig bronnen toe te voegen? Bijvoorbeeld Zotero (\href{https://www.zotero.org/}{zotero.org}), de mensen die dit eenmaal hebben gebruikt zweren erbij.
	\item Wist je dat in de Russische literatuur de integraaltekens net als de communisten naar links leunen? Hoe je dit bereikt in \LaTeX\ is te vinden op \href{https://latex.org/forum/viewtopic.php?t=23928}{latex.org/forum/viewtopic.php?t=23928}.
\end{itemize}

\printbibliography

\end{document}
