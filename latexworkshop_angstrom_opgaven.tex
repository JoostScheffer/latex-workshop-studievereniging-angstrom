\documentclass{article}

\usepackage[english, dutch]{babel}
\usepackage[style=ieee, backref=true, backend=biber]{biblatex}
\addbibresource{referenties.bib} % Importeert het bibliografiebestand

\usepackage{parskip}
\usepackage{doc}
\usepackage{amsmath, amssymb}
\usepackage{color}
\usepackage{qrcode}
\usepackage{graphicx} 
\usepackage{bookmark} 
\usepackage{enumerate}
\usepackage{float}
\usepackage{listings}
\lstset{language=C++}
\usepackage[per-mode=reciprocal, separate-uncertainty=true]{siunitx}
\addto\extrasgerman{\sisetup{locale = DE}}
\sisetup{%
	output-decimal-marker = {,},
	inter-unit-product = \ensuremath{{}\cdot{}},
	exponent-product = \ensuremath{{}\cdot{}},
	list-final-separator = { en },
	list-pair-separator = { en },
	range-phrase = { tot },
}

\usepackage{hyperref}
\hypersetup{
  colorlinks   = true,
  urlcolor     = blue,
  linkcolor    = blue,
  citecolor   = purple,
  bookmarksopen = true,
  bookmarksopenlevel = 2,
}
\usepackage{url}
\usepackage{metalogo}

\usepackage{tabularray}
\usepackage{caption}
\captionsetup{labelfont={bf}}
\definecolor{bg}{rgb}{0.95,0.95,0.95}

\usepackage{minted} 
\setminted[latex]{obeytabs=true, tabsize=2, breaklines, framesep=5mm, bgcolor=bg}
\setminted[bibtex]{obeytabs=true, tabsize=2, breaklines, framesep=5mm, bgcolor=bg}
\setminted[text]{obeytabs=true, tabsize=2, breaklines, framesep=5mm}
% \usemintedstyle{bw}

\usepackage[a4paper,left=3cm,right=2.5cm,top=1.5cm,bottom=1.5cm,heightrounded]{geometry}

% \usepackage{tcolorbox}
% \newtcolorbox{mybox_1}{colback=blue!5!white,colframe=blue!75!black}

\usepackage{csquotes}

\title{\LaTeX-workshop voor beginners (Opgaven)}
\author{Joost Scheffer, studievereniging Ångström}
\date{\today}
\begin{document}
\maketitle
\section{Nieuw document}
\begin{enumerate}
	\item Begin een \LaTeX-document met de tekst ``Hello World!''
\end{enumerate}
\section{Tekst}
\begin{enumerate}
	\item Maak twee paragrafen en twee deelparagrafen, zoek op het internet op hoe je een ongenummerde paragraaf maakt.
	\item Zoek op (op internet) hoe je accenten kunt maken in letters, zoals bij coördinaat, café, curaçao\"enaar, etc. Kijk ook naar het verschil tussen ''tekst'' en ``tekst'' (let op de accentjes).
	\item  Sommige karakters, zoals \verb|{| hebben al een betekenis binnen \LaTeX. Hoe denk je dat je deze karakters in een tekst zou kunnen weergeven in de PDF. Tip: hoe zien de commando’s er standaard uit?
	\item Maak zelf een paar nieuwe commando’s. Dit doe je door \verb|\newcommand{}{}| in de preamble te zetten. Het eerste argument is de naam van je commando (bijvoorbeeld \verb|\R|), en het tweede argument is het commando dat daardoor uitgevoerd moet worden (bijvoorbeeld \verb|\mathbb{R}|).
	\item Zoek op het internet op hoe je een bereik aangeeft met een half-kastlijntje (\textit{en-dash}) zoals in ``20--26''.
\end{enumerate}
\section{Formules en getallen}
\begin{enumerate}
	\item Reproduceer de volgende formules. Let op de accolades!
	      \begin{enumerate}
		      \item \[ a_{1,1} + a_{1,2} + \ldots + a_{1,n} = \sum_{i=1}^{n} a_{1,i} \]
		      \item \[ \lim_{\xi \to \infty} 2^{-\xi} = 0 \]
		      \item \[ \begin{pmatrix} k\\n  \end{pmatrix} = \prod_{\ell=1}^{n} \frac{k-\ell+1 }{\ell} \]
		      \item \[ \Omega[n] \equiv \sin^2\left( \frac{\pi n}{M - 1} \right) \]
	      \end{enumerate}
	\item Zorg er voor dat een van de vergelijkingen een nummer krijgt.
	\item Gebruik de commando's uit het \texttt{siunitx} package en maak het volgende na
	      \begin{enumerate}
		      \item De straal van de aarde is \qty{6 357}{\kilo\metre}
		      \item De straal van de aarde is \qty{6.357e6}{\metre}
		      \item \( h = \qty{6.62607015e-34}{\joule\per\hertz} \)
		      \item \( q = \qty{5+-0.01e3}{\coulomb} \)
	      \end{enumerate}

\end{enumerate}

\newpage
\section{Afbeeldingen invoegen}
Begin deze paragraaf op een nieuwe pagina.
Voeg een afbeelding toe aan de bovenkant van de pagina, met onderschrift, zoals hierboven. Plaats ook een afbeelding in het midden in een stuk tekst.

\begin{figure}[t]
	\begin{center}
		\includegraphics[width=0.3\linewidth]{logo.pdf}
	\end{center}
	\caption{Ångström logo}
\end{figure}

\section{Referenties}
\begin{enumerate}
	\item Kies je favoriete vergelijking uit Paragraaf 3 en maak daar een verwijzing naar.
	\item Maak twee keer een bibliografie ``entry'' en citeer deze.
\end{enumerate}

\end{document}
